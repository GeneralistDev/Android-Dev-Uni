% Software Development for Mobile Devices
\documentclass[11pt,english,numbers=endperiod,parskip=half,abstract=on]{scrartcl}

\usepackage{color}
\usepackage{graphicx}
\usepackage{minted}
\usepackage{fancyhdr}
\usepackage{pdflscape}
\usepackage{listings}
\usepackage{pifont}
\usepackage{pdfpages}
\usepackage{hyperref}
\usepackage{subcaption}

\newcommand{\cmark}{\ding{51}}

\pagestyle{fancy}

\rhead{Daniel Parker - 971328X}
\lhead{COS30017 - Software Development for Mobile Devices}

\title{HD Research Report}
\subtitle{COS30017 - Software Development for Mobile Devices}
\author{Daniel Parker 971328X}

\date{\today}

\begin{document}
\maketitle
\begin{abstract}
  This research report seeks to show the findings of using the RAPPT Android
  prototyping tool as an initial design and construction tool for Android app
  development. More specifically it hopes to prove that the use of such a
  prototyping tool can significally shorten the development timeframe of an app
  from planning to alpha-level executable. The report also covers any pitfalls
  of using the tool and tries to identify areas of improvement for the tool to
  increase it's viability in mainstream application development.
\end{abstract}
\thispagestyle{empty}

\section{Introduction}

\section{Method}
  The goal for this research method was to closely follow a 38 hour week
  proper development cycle, as one would expect a professional developer to
  do in the commercial environment. The 38 hour weeks are worked by a single
  developer only.
  \begin{enumerate}
    \item{
      Conception of app idea happens prior to the timed process as it isn't
      taken to be an important factor into this study.
    }
    \item{
      The date that planning and design begins on is recorded. The time that
      this takes to complete is also insignificant when assessing the prototyping
      tool, however the importance of this step occurring is paramount due to it
      laying the foundation for the developer to continue smoothly onto the
      prototyping stage and not mistakenly label the planning and design stage
      as part of the prototyping stage, which it is not.
    }
    \item{
      Once planning and designs are complete, the date is recorded as the
      date that prototyping begins.
    }
    \item{
      The app is prototyped using RAPPT as many times as needed until the
      developer feels they have a base from which they have achieved all they
      can using the prototyping tool. In other words, if the prototyping tool
      cannot implement anymore of the features or layouts of the app then
      the prototyping should cease. The date is recorded for when this occurs.
    }
    \item{
      Record the date as the start of extending / non-prototype development.
      The developer should prioritise the main features of the app above other
      aspects and ensure that they are implemented on
      top of the codebase supplied by RAPPT. Make notes of the areas of the app
      that were made easier to develop on by the prototype and those areas
      that were more difficult to continue developing on.
    }
    \item{
      Once all major / critical components of the app are functional and stable
      the developer should improve the visual, usability and overall stability
      of the app until they feel it has reached an alpha release level. That is,
      there may be bugs present, however the main app features are functional
      and should be usable by initial-uptakers / testers. Record this date
      as the termination of alpha development.
    }
    \item{
      Run `lines of code' counter on the original prototype and the final alpha
      source code to see how much was generated vs. how much was an extension
      of the prototype. Estimate based on lines of code written and time taken
      how much time was saved by the prototype tool.
    }
  \end{enumerate}

\section{Results}

\section{Discussion}

\section{Conclusion}

\section{References}
  \begin{enumerate}
    \item{

    }
    \item{

    }
    \item{

    }
    \item{

    }
  \end{enumerate}
\end{document}
