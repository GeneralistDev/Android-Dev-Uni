\documentclass[a4paper, parskip=half]{scrartcl}
\makeatletter
\renewcommand\@seccntformat[1]{}
\makeatother

\title{Usability Test Information}
\author{Daniel Parker - 971328X}
\date{}
\begin{document}
\maketitle

\section{About Suntime}
\subsection{Idea}
An app. that will show the Sun Rise, Set time and weather forecast at a given
location for any valid date (past/present/future).

\subsection{Motivation}
This app. will provide useful information for photographers, bush walkers, and
people that undertake prayers/spiritual practices based on the sun rise/set
times.

\subsection{Key Features}
\begin{itemize}
  \item{Show sun rise/set times for a date/location.}
  \item{Can add new custom locations (or) select from pre-built set of locations.}
  \item{Generate a table of sun rise/set times for a date range.}
  \item{Share information via SMS and email.}
  \item{Can detect current location.}
  \item{Integrated into Google maps.}
  \item{Can detect current location.}
  \item{View sun rise/set times for various locations on a map.}
  \item{View weather forecast (current, and near future)}
\end{itemize}

\subsection{Scenarios}
\begin{enumerate}
  \item{
    Brad is planning a short 3 day holiday in Wellington, NZ (travelling
    next month). He wants to take a few photographs of the sun set over the
    harbour and wants to make sure his flight times give him sufficient
    opportunities to take these pictures.
  }

  \item{
    Sachin has to undertake a religious fast for 40 days from sun rise to
    sun set start- ing in mid-May. Unfortunately, he is travelling during this
    time to 3 different countries across the world (China, US and India). Sachin
    works for a large mining company and the locations that he is travelling to
    are very remote placed in these countries. He generates a table of sun
    rise/set times for each of his locations, emails them and print the email
    message ahead of this journey time.
  }

  \item{
    Li wants to walk on the beach tomorrow morning to reflect on the purpose
    of life (she was just promoted in her job). She checks the sun rise time in
    Sydney be- fore going to bed.
  }

  \item{
    Justin and Mary are off camping. They reach the camp site and realise
    that they are a little bit behind schedule. They need to start off at day
    break to get to the top of the mountain as planned. They use the built-in
    GPS facility to find the sun rise/set times for their location. As they have
    a faint mobile signal, they send the sun-rise time to their friends that are
    also climbing the mountain from another direction. They add a short note to
    the message saying they are looking forward to beating them to the top of the
    mountain.
  }
\end{enumerate}

\section{Usability Test Information}
You will need to perform the above scenarios as best you can on the prototype
application. Note that the prototype is limited in functionality. Read out loud
the scenario before you attempt it, and inform the supervisor when you are going to
begin to attempt the task.


The test supervisor will not be able to assist you in using the app as this would skew
any results of the test, however you should let the supervisor know if you are stuck.
The supervisor is able to act as the phone's GPS sensor as the prototype is lacking
in that functionality.


\textbf{Whilst attempting the scenarios, please verbalise your thoughts in
regards to what you are doing as this will assist the tester in their study.}

At the end of the usability test, you will be required to answer a short survey
which the supervisor will provide to you.

\textbf{The supervisor will be recording your comments on paper for reference. Your personal
details will remain confidential, and any information retained will be kept
separate from your name, and will not be attributed to you in any reports.}

\end{document}
